% 使用国内主流的 exam-zh 试卷类
\documentclass{exam-zh}

% --- 试卷宏包配置 ---
\examsetup{question/show-answer = true}

% 引入 enumitem 以便自定义层级序号 (1) -> (i) -> (a)
\usepackage{enumitem}

\begin{document}

% --- 定义递归渲染宏 (Macro) ---
% 这里的逻辑是:如果有 sub_questions,就生成 enumerate 环境,并递归调用
((* macro render_sub_questions(sub_list, level=1) *))
    % 根据层级设定不同的标签样式
    % 第一层用 (1), (2); 第二层用 (i), (ii); 第三层用 (a), (b)
    ((* if level == 1 *))
    \begin{enumerate}[label=(\arabic*)]
    ((* elif level == 2 *))
    \begin{enumerate}[label=(\roman*)]
    ((* else *))
    \begin{enumerate}[label=(\alph*)]
    ((* endif *))

    ((* for sub in sub_list *))
        \item (( sub.content ))
        % 递归检查:如果当前小问还有子小问,递归调用宏,层级+1
        ((* if sub.sub_questions *))
            (( render_sub_questions(sub.sub_questions, level + 1) ))
        ((* endif *))
    ((* endfor *))

    \end{enumerate}
((* endmacro *))
% -----------------------------

% --- 试卷表头 ---
\title{ (( meta.title )) }
\subject{ (( meta.subject )) }
\maketitle


% --- 循环遍历所有大题 ---
((* for section in sections *))

\section*{ (( section.title )) }

    % === 类型1:选择题 (单选 & 多选) ===
    ((* if section.type == 'single_choice' or section.type == 'multiple_choice' *))
        ((* for q in section.questions *))
        \begin{question}
            (( q.content )) \paren[]

             % TikZ 图形渲染
            ((* if q.figure and q.figure.type == 'tikz' *))
            \begin{center}
                (( q.figure.code ))
            \end{center}
            ((* endif *))

            \begin{choices}
                ((* for opt in q.options *))
                \item (( opt ))
                ((* endfor *))
            \end{choices}
        \end{question}
        ((* endfor *))

    % === 类型2:填空题 ===
    ((* elif section.type == 'fill' *))
        ((* for q in section.questions *))
        \begin{question}
            ((* if q.figure and q.figure.type == 'tikz' *))
            \begin{center}
                (( q.figure.code ))
            \end{center}
            ((* endif *))
            (( q.content ))
        \end{question}
        ((* endfor *))

    % === 类型3:解答题 (重点修改部分) ===
    ((* elif section.type == 'problem' *))
        ((* for q in section.questions *))
        \begin{problem}[points = (( q.score | default(10) )) ]
            (( q.content ))

            ((* if q.figure and q.figure.type == 'tikz' *))
            \begin{center}
                (( q.figure.code ))
            \end{center}
            ((* endif *))

            % 调用上面定义的递归宏
            ((* if q.sub_questions *))
                (( render_sub_questions(q.sub_questions, 1) ))
            ((* endif *))

            % 预留空白供作答
            \vspace{8cm} 
        \end{problem}
        ((* endfor *))

    ((* endif *))

((* endfor *))

\end{document}
